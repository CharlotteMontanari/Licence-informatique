\documentclass[hyperref={pdfpagemode=FullScreen,colorlinks=false}]{beamer}
%\documentclass[aspectratio=169,hyperref={pdfpagemode=FullScreen,colorlinks=false}]{beamer}
\usepackage{concrete}
\usepackage[utf8]{inputenc}
\usepackage[T1]{fontenc}
\usepackage[french]{babel}
\usetheme{Madrid}
\usetheme{Warsaw}
\uselanguage{French}
\languagepath{French}
\title{Le titre de la présentation}
\author{G. \textsc{Faccanoni}}
\institute{IMATH-UTLN}

\begin{document}
\begin{frame}[plain]
\maketitle
\end{frame}
\begin{frame}
Le texte de ma diapo.
\end{frame}
\begin{frame}
\frametitle{Titre de la diapo}
Bla bla
\begin{definition}
Le texte de la définition.
\end{definition}
Bla bla
\begin{example}
Le texte de l'exemple.
\end{example}
Bla bla
\begin{theorem}
Le texte du théorème.
\end{theorem}
\begin{proof}
Le texte de la démonstration.
\end{proof}
\end{frame}

\begin{frame}
\frametitle{Titre de la diapo}
\begin{itemize}
\item item 1
\begin{itemize}
\item item 1.1
\item item 1.2
\begin{itemize}
\item item 1.2.1
\item item 1.2.2
\end{itemize}
\end{itemize}
\item item 2
\end{itemize}
\end{frame}



\end{document}



