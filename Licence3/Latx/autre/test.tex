\documentclass[a4paper,10pt]{scrartcl}
\usepackage[utf8]{inputenc}
\usepackage[T1]{fontenc}
\usepackage[french]{babel}
\usepackage[dvipsnames,svgnames]{xcolor}
\usepackage{hyperref}\hypersetup{colorlinks=true,linkcolor=Brown,citecolor=ForestGreen}
\usepackage{multicol}
\usepackage{lipsum}
\usepackage{xspace}
\usepackage{mathtools, amssymb}
\usepackage{amsthm}
\newcommand{\R}{\mathbb{R}}
\newcommand{\abs}[1]{\left\lvert#1\right\rvert}
\newtheorem{theoreme}{Théorème}[section]
\newcommand{\card}[1]{\left\lvert#1\right\rvert}
\newcommand{\ensnombre}[1]{\mathbb{#1}}
\newcommand{\N}{\ensnombre{N}}
\newcommand{\Z}{\ensnombre{Z}}
\newcommand{\sphereunite}[1]{\ensnombre{S}^{#1}}
\newcommand{\norme}[1]{\left\lVert#1\right\rVert}
\newcommand{\coeffbinom}[2]{C_{#1}^{#2}}
\newcommand{\diff}{\,\mathrm{d}}
\newcommand{\mtext}[1]{\quad\text{#1}\quad}


\begin{document}
\section{Introduction}
bla bla bla
\subsection{Premier Essai}\label{subsec.toto}
bla bla bla
\subsection{Deuxième Essai}
bla bla bla
On a vu à la section~\ref{subsec.toto} à la page~\pageref{subsec.toto}

\begin{center}
\begin{tabular}{|clr|}
\hline
bla bla & bla bla bla & bla       \\
\hline
\multicolumn{2}{|c}{fusionnées} & bla bla \\
\hline
bla bla & bla bla bla & bla       \\
\hline
\end{tabular}
\end{center}

%https://www.tablesgenerator.com/ pour les tableaux
\begin{tabular}{|r|c|l|}
\hline
AA  &   BBB &   C\\
\hline
D   &   EE  &   FFF\\
GGG &   H   &   II\\
\hline
\end{tabular}

\begin{multicols}{2}
\lipsum[1]
\begin{multicols}{2}
\lipsum[2]
\end{multicols}
\lipsum[3]
\end{multicols}

Soit $f$ une fonction vérifiant
\begin{equation}\label{eq.fonction.f}
f(x) = 2x + 1    
\end{equation}
On a $f(x) - 1 = 2x$ d'après la formule \eqref{eq.fonction.f}

\[
    (x^2)^3 = x^{2^3}\]
\[
    F_n = 7^{2^n} + 1\]

\[
    u_{n+1} = \sqrt[n]{1+u_n}
    \]

\[
    x_{5} = \sqrt{1+\sqrt{2+\sqrt{3+\sqrt{4+\sqrt{5}}}}}
    \]

\[
    (\sqrt{x})^2 = x \text{ mais }  \sqrt{x^2} \neq x   \text{ en général.}
    \]

\[
    y=x^2 \iff x=y^{1/2}
    \]

\[
    x>0 \implies x^2\neq0
    \]

\[
    x\in X\setminus Y \implies x \not\in Y
    \]

\[
    x^{1/3} = x^{\frac{1}{3}} = \sqrt[3]{x}
    \]

\[
    \sqrt{2} = 1 + \frac{1}{2+\frac{1}{2+\frac{1}{2+\frac{1}{\ddots}}}}
    \]
\[
    \frac{\pi^2}{6} + \gamma = \Gamma(n) + \sqrt[n]{1+\alpha}
    \]
\[
    \cos^2(x) + \sin^2(x) = 1
    \]
\[
    2^{\ln(x)} = x^{\ln(2)}
    \]
\[
    \sum_{n=1}^{+\infty} \frac{1}{n^2} = \frac{\pi^2}{6}
    \]
\[
    \int_0^1 -\frac{\ln(1-t)}{t} dt \approx 1{,}64493
    \]
\[
    \max_{  \substack{x,y \in E \\ x \cdot y = 0}  } \varphi(x)
    \]
\[
    \overrightarrow{OM}
    =
    \underbrace{O + \vec{u}}_{\text{point}+\text{vecteur}}
    \]
\[
    \lVert x \rVert = 1 \iff \langle x, x \rangle = 1
    \]
\[
    \lvert\{1, 2, \dots, n\}\rvert = n
    \]
\[
    \lfloor x^2+\epsilon \rfloor = \lceil \sqrt{y}+\delta \rceil
    \]
\[
    \left\lfloor\sum_{n=1}^{N}{u_n}\right\rfloor^2 = N^2+N+1
    \]
\[
    \left[1+\left(\int_0^{\sqrt{2}} f\right)^2\right] = \gamma
    \]
\[
    \left.
    \begin{array}{r}
    a\in\mathbb{C}\\
    a\not\in\mathbb{R}
    \end{array}
    \right]
    \implies
    a\in\mathbb{C}\setminus\mathbb{R}
    \]
\[
    \mathbb{M} =
    \begin{pmatrix}
    m_{1,1} & \dots     & m_{1,n} \\
    \vdots  & \ddots    & \vdots \\
    m_{n,1} & \dots     & m_{n,n}
    \end{pmatrix}
    \]
\begin{align*}
    f\colon \R  & \to \R        & g\colon \R & \to \R \\
    x           & \mapsto x^2   & x          & \mapsto \sqrt{x}
    \end{align*}
\begin{align*}
    \abs{\int_{a}^{b}{(f+g)}}
        & =    \abs{\int_{a}^{b}{f}  + \int_{a}^{b}{g}}      \\
        & \leq \abs{\int_{a}^{b}{f}} + \abs{\int_{a}^{b}{g}} \\
        & \leq \int_{a}^{b}{\abs{f}} + \int_{a}^{b}{\abs{g}}
    \end{align*}
\[
    \sqrt{\sum_{n=0}^{+\infty} u_n}
    =
    \left(\int_a^b f \,\mathrm{d}t\right)^2 + \gamma \times \frac{\pi^2}{6}
    \]
\[
    \lim_{x \to 0^+} f(x) = \sin \left|\frac{1-\sqrt[2]{3}}{2}\right|
    \quad \text{mais} \quad f(0) = \ln^2 (3)
    \]
\[
    \lim_{\substack{x \to 0 \\ x > 0}}
    \frac{ \left\lfloor(\sin (x))^{\cos (x)}-(\cos x)^{\sin (x)}\right\rfloor }{x^3}
    \]
\[
    \forall x \in \mathbb{R}, \quad f(x) = 0 \iff x^{1/3} + \ln(\tan(x)) = \omega
    \]

Voici un certain nombre d'ouvrages utiles. Le plus simple d'accès est~\cite{lcfr}.
Pour de nombreuses autres références utiles, on pourra consulter~\cite[p.~3]{El03}.
\bibliographystyle{plain}
\bibliography{biblio}

\end{document}